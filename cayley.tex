\documentclass{unswmaths}

\usepackage{unswshortcuts}

\begin{document}

\subject{}
\author{}
\title{}
\studentno{}


\newcommand{\Real}{\operatorname{Re}}
\newcommand{\Img}{\operatorname{Im}}
\newcommand{\lan}{\langle}
\newcommand{\ran}{\rangle}
\newcommand{\Proj}{\mathbb{P}}
\newcommand{\isom}{\cong}
\newcommand{\id}{{\operatorname{id}}}
\newcommand{\ha}{\boldsymbol{m}}
\newcommand{\Circ}{\mathbb{T}}
\newcommand{\BMO}{{BMO}}
\newcommand{\sgn}{\operatorname{sgn}}
\newcommand{\Diff}{\mathcal{D}}
\newcommand{\pvint}{\mathrm{p.v.}\int}
\newcommand{\Half}{\mathbb{H}}
\newcommand{\Disc}{\mathbb{D}}


\section*{Introduction}
The purpose of these notes is to describe the relationship between
function spaces on the circle and on the line by the Cayley transform.

\section*{Notation}
$\Half = \{z \in \Cplx\;:\;\Img(z) > 0\}$ is the upper half plane,
and $\Disc = \{z \in \Cplx\;:\;|z| < 1\}$ is the unit disc. $\ha$
denotes the normalised Haar measure on $\Circ$, and $\lambda$ is the Lebesgue measure 
on $\Rl$. $\ha_2$ denotes the area measure on $\Cplx$.


$E$ denotes a Banach space. By $L^p(X;E)$, we mean
\begin{equation*}
    \|f\|_{L^p(X;E)}^p = \int_X \|f\|_E^p\;d\mu
\end{equation*}
for $0 < p < \infty$, and we say $f \in L^\infty(X;E)$ when $\|f\|_E \in L^\infty(X)$. 

\section*{The Cayley Transform}
The Cayley transform is a conformal mapping of the complex plane to itself, given by
\begin{equation*}
    \omega: z\mapsto \frac{z-i}{z+i}.
\end{equation*}
Defined for $z \neq -i$. In particular, $\omega$ maps the upper half plane to the unit disc, 
and the real line to the unit circle with $1$ removed. Note that
\begin{equation*}
    \omega^{-1}: z\mapsto -i\frac{z+1}{z-1}.
\end{equation*}


Since this is not a bijection, we
must make careful note of the following technical point:

\begin{remark}
    Throughout these notes, as is typical in analysis, functions on $\Rl$ or $\Circ$
    are defined only up to almost everywhere equivalence. The spaces of almost-everywhere
    equivalence classes of measurable $E$-valued functions on a measure space $X$
    is denoted $L^0(X;E)$. 
    
    If $X$ is a topological space, we consider $C(X;E) \subseteq L^0(X;E)$ by identifying
    a continuous function with an equivalence class of functions that agree with it almost everywhere.
\end{remark}

Given $f \in L^0(\Rl;E)$, we define
\begin{equation*}
    \mathcal{U}f := f\circ \omega^{-1} \in L^0(\Circ;E).
\end{equation*}
and for $g \in L^0(\Circ;E)$
\begin{equation*}
    \mathcal{U}^{-1} g = g\circ \omega.
\end{equation*}

It is evident that $\mathcal{U}$ is a well defined linear isomorphism of $L^0(\Rl;E)$ to $L^0(\Circ;E)$. 

\section*{Integrability and the Cayley transform}
The first result of use is,
\begin{lemma}
    For $0 < p \leq \infty$, $U$ maps continuously from $L^p(\Rl)$ to $L^p(\Circ)$. 
\end{lemma}
\begin{proof}
    Let $f \in L^p(\Rl)$. Then we simply compute,
    \begin{align*}
        \|\mathcal{U}f\|_p^p &= \int_\Circ \left|f\left(-i\frac{z+1}{z-1}\right)\right|^p d\ha(z)\\
        &= \frac{1}{2\pi} \int_0^{2\pi} 
    \end{align*}
    blah blah.
\end{proof}

A natural question is to ask for which $p$ is $\mathcal{U}L^p(\Rl) = L^p(\Circ)$? It is obvious
that this is true for $p = \infty$, less obvious is that this works for $p = 2$.

\section*{Hardy spaces}
Hardy spaces on $\Disc$ as defined as spaces of holomorphic functions as follows,
\begin{definition}
    Let $0 < p < \infty$. Given a holomorphic $f:\Disc\rightarrow E$, we define
    \begin{equation*}
        \|f\|_{H^p(\Disc;E)}^p = \sup_{0 < r < 1} \left(\int_\Circ \|f(rz)\|_E^p \;d\ha(z)\right).
    \end{equation*}

    
    Similarly, define
    \begin{equation*}
        \|f\|_{H^\infty(\Disc;E)} = \sup_{z \in \Disc} \|f(z)\|_E.
    \end{equation*}
    
    We say that $f \in H^p(\Disc;E)$ if $\|f\|_{H^p(\Disc;E)} < \infty$ where $0 < p \leq \infty$.
\end{definition}

There are analogous Hardy spaces on the upper half plane,
\begin{definition}
    Let $0 < p < \infty$. Given a holomorphic $f:\Half\rightarrow E$, we define
    \begin{equation*}
        \|f\|_{H^p(\Half;E)}^p = \sup_{y > 0} \left(\int_\Rl \|f(x+iy)\|_E^p\;d\lambda(x) \right).
    \end{equation*}
    Similarly, define
    \begin{equation*}
        \|f\|_{H^\infty(\Half;E)} = \sup_{z \in \Half} \|f(z)\|_E
    \end{equation*}
   
    We say that $f \in H^p(\Half;E)$ if and only if $\|f\|_{H^p(\Disc;E)} < \infty$ for $0 < p \leq \infty$
\end{definition}

The Hardy spaces are defined as spaces of functions on $\Disc$ and $\Half$. However, they are canonically
identified with spaces of functions on $\Disc$ and $\Rl$ by taking ``boundary values". This is justified
by the following lemma,
\begin{lemma}
    For $p \in (0,\infty]$, let $f \in H^p(\Disc;E)$. Then for almost all $z \in \Circ$, the limit
    \begin{equation*}
        \tilde{f}(z) := \lim_{r\rightarrow 1} f(rz)
    \end{equation*}
    exists, and defines $\tilde{f} \in L^p(\Circ;E)$. If $g \in L^0(\Circ;E)$
    such that $g = \tilde{f}$ for some $f \in H^p(\Disc;E)$, we say that $f \in H^p(\Circ;E)$. 
\end{lemma}

\begin{thebibliography}{9}
\bibitem{connes94}
     Connes A., 
    \emph{Noncommutative Geometry}
     Academic Press, 
     San Diego, 
     CA, 
     1994
\bibitem{peller}
    Peller V.V.,
    \emph{Hankel Operators and their Applications}
    Springer-Verlag,
    New York, 
    NY,
    2003    
\bibitem{garnett}
    Garnett J.B.
    \emph{Bounded analytic functions}
    Springer-Verlag,
    New York,
    NY,
    2007    
\end{thebibliography}


\end{document}
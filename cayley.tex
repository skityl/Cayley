\documentclass{unswmaths}

\usepackage{unswshortcuts}

\begin{document}

\subject{}
\author{}
\title{}
\studentno{}


\newcommand{\Real}{\operatorname{Re}}
\newcommand{\Img}{\operatorname{Im}}
\newcommand{\lan}{\langle}
\newcommand{\ran}{\rangle}
\newcommand{\Proj}{\mathbb{P}}
\newcommand{\isom}{\cong}
\newcommand{\id}{{\operatorname{id}}}
\newcommand{\ha}{\boldsymbol{m}}
\newcommand{\Circ}{\mathbb{T}}
\newcommand{\BMO}{{BMO}}
\newcommand{\sgn}{\operatorname{sgn}}
\newcommand{\Diff}{\mathcal{D}}
\newcommand{\pvint}{\mathrm{p.v.}\int}
\newcommand{\Half}{\mathbb{H}}
\newcommand{\Disc}{\mathbb{D}}
\newcommand{\U}{\mathcal{U}}


\section*{Introduction}
The purpose of these notes is to describe the relationship between
function spaces on the circle and on the line by the Cayley transform.

\subsection*{Notation}
$\Half = \{z \in \Cplx\;:\;\Im(z) > 0\}$ denotes the upper half plane,
and $\Disc = \{z\in \Cplx\;:\; |z| < 1\}$ denotes the open unit ball.
$\Circ = \{z \in \Cplx\;:\; |z| = 1\}$.

We use normalised Haar measure on $\Circ$, denoted $\ha$. Lebesgue
measure on $\Rl$ is denoted $\lambda$, and two dimensional Lebesgue measure on
$\Cplx$ is denoted $\ha_2$.

Throughout these notes, $\omega$ denotes the \emph{Cayley transform}.
$\omega:\Cplx\setminus\{1\}\rightarrow \Cplx$, and 
\begin{equation*}
    \omega(\zeta) = i\frac{1+\zeta}{1-\zeta},\;\zeta \in \Disc.
\end{equation*}

For a Banach space $E$, and a measure space $(X,\Sigma,\mu)$, we define
\begin{equation*}
    \|f\|_{L^p(X;E)} = \left(\int_X \|f\|_E^p \;d\mu\right)^{1/p}
\end{equation*}
for $p \in (0,\infty)$, and
\begin{equation*}
    \|f\|_{L^\infty(X;E)} = \sup_{x \in X} \|f(x)\|_E
\end{equation*}
for a weakly measurable $f:X\rightarrow E$. We define $L^p(X;E)$ as the set
of measurable $f:X\rightarrow E$ with $\|f\|_{L^p(X;E)} < \infty$. As usual, 
we identify together functions on a measure space $(X,\Sigma,\mu)$ 
which agree $\mu$-almost everywhere.

$L^0(X;E)$ denotes the set of all ($\mu$-almost everywhere equivalence classes of)
weakly measurable functions from $X$ to $E$.

When $X$ is a set with counting measure, we denote $L^p(X;E)$ as $\ell^p(X;E)$.

Suppose $\zeta \in \Circ$. Provided that $\zeta \neq 1$, we see that $\omega(\zeta)$
is defined, and $\omega$ maps $\Circ\setminus\{1\}$ smoothly to $\Rl$. Thus for
$f \in L^0(\Rl;E)$, we can define $\tilde{f} \in L^0(\Circ;E)$
by 
\begin{equation*}
    \tilde{f} := f\circ \omega^{-1}.
\end{equation*}

Thus we can define the important operator $\U:L^0(\Circ;E)\rightarrow L^0(\Rl;E)$,
\begin{equation*}
    (\U f)(x) = \frac{1}{\sqrt{\pi}}\frac{(f\circ \omega^{-1})(x)}{x+i}.
\end{equation*}

It is the purpose of these notes to collate various results concerning the images
of certain subspaces of $L^0(\Circ;E)$ under $\U$.

\subsection*{Fourier transform}
Let $(X,\Sigma,\mu)$ be a measure space. By assumption, $f \in L^0(X;E)$ is 
weakly measurable. If we make the further assumption that $f$ is \emph{almost surely
separably valued}, that is, there is some $N \subset X$ with $\mu(N) = 0$
such that $f(X\setminus N) \subseteq E$ is separable, then we 
may define the Bochner integral,
\begin{equation*}
    \int_X f\;d\mu.
\end{equation*}

Suppose that $f \in L^1(\Circ;E)$ is almost surely separably valued. Then
for each $n \in \Intgr$, the integral
\begin{equation*}
    \hat{f}(n) = \int_\Circ f(\zeta)\zeta^{-n}\;d\ha(\zeta)
\end{equation*}
exists. 
\begin{theorem}
    Let $E$ be separable.
    The map
    \begin{equation*}
        f \in L^2(\Circ;E)\mapsto \{\hat{f}(n)\}_{n\in\Intgr}
    \end{equation*}
    is an isometric isomorphism of Banach spaces,
    \begin{equation*}
        L^2(\Circ;E) \isom \ell^2(\Intgr;E).
    \end{equation*}
    When $E$ is a Hilbert space, this is an isometric isomorphism
    of Hilbert spaces.
\end{theorem}       

\subsection*{Spaces of smooth functions}
The spaces $C(\Circ;E)$ and $C(\Rl;E)$ denote spaces of continuous $E$-valued functions on
the circle and line respectively.

Since $C(\Circ;E) \subseteq L^2(\Circ;E)$, we may define for $f \in C(\Circ;E)$
and $s \in \Cplx$ with $\Re(s) > 0$. 
\begin{equation*}
    I_s(f) = \sum_{j \in \Intgr} \frac{1}{(1+|j|)^s}\hat{f}(j)z^j.
\end{equation*}
$I_s$ is the fractional integral operator of order $s$, also called a Bessel potential.
We define
\begin{equation*}
    C^s(\Circ;E) = I_sC(\Circ;E).
\end{equation*}


\section*{Initial results}
It is obvious from the definition that $\U$ is linear. 

Now define, for $g \in L^0(\Rl;E)$,
\begin{equation*}
    (\mathcal{F})g(\zeta) = \frac{\sqrt{\pi}}{2i}\frac{(g\circ \omega)(\zeta)}{1-\zeta},\;\zeta\in\Circ.
\end{equation*}

\begin{lemma}
    $\U$ and $\mathcal{F}$ are inverse functions, hence $\U$ is a bijection.
\end{lemma}
\begin{proof}
    Let $g \in L^0(\Rl;E)$, and let $t \in \Rl$. Then we simply compute,
    \begin{align*}
        (\U\circ\mathcal{F})(g)(t) &= \frac{\sqrt{\pi}}{2i}\frac{g(t)}{1-\omega^{-1}(t)} \frac{1}{\sqrt{\pi}}\frac{1}{t+i}\\
        &= \frac{1}{2i(1-\omega^{-1}(t))(t+i)}g(t)\\
        &= g(t).
    \end{align*}
    So $\U\circ \mathcal{F}$ is the identity function on $L^0(\Rl)$.
    
    Similarly, let $f \in L^0(\Rl;E)$, and $\zeta \in \Circ$, then
    \begin{align*}
        (\mathcal{F}\circ\U)(f)(\zeta) &= \frac{\sqrt{\pi}}{2i}\frac{1}{1-\zeta}\frac{g(\zeta)}{i+\omega(\zeta)}\\
        &= g(\zeta).
    \end{align*}
    
    So $\mathcal{F}\circ\U$ is the identity function on $L^0(\Circ;E)$.
\end{proof}

$\omega$ may be regarded as a function from $\Circ\setminus\{1\}\rightarrow \Rl$. 
If we define $\omega(1)$ to be some arbitrary value, say $\omega(1) = 0$, we
have a measureable function $\omega:\Circ\rightarrow\Rl$.

Thus there is a pushforward of the Haar measure $\ha$ on $\Circ$
to $\Rl$, denoted $\omega_*(\ha)$, defined
by
\begin{equation*}
    \omega_*(\ha)(A) = \ha(\omega^{-1}(A))    
\end{equation*}
for all Lebesgue measurable sets $A$.


We may describe this with the following result,
\begin{lemma}
    The pushforward measure, $\omega_*(\ha)$ has Lebesgue Radon-Nikodym 
    derivative
    \begin{equation*}
        \frac{d\omega_*(\ha)}{d\lambda} = \frac{1}{\pi|i+t|^2}
    \end{equation*}
\end{lemma}
\begin{proof}
    Let $a$ be the arc length measure on $\Circ$, so $\ha = \frac{a}{2\pi}$, now
    let $A \subseteq \Rl$ be lebesgue measurable, then
    \begin{align*}
        \omega_*(a)(A) &= \int_A \left|\frac{d(\omega^{-1}(t)}{dt}\right|\;d\lambda(t)\\
        &= \int_A \frac{2}{|i+t|^2}d\lambda(t).
    \end{align*}
    Hence, the required result follows.
\end{proof}


A less obvious result is the following, 
\begin{theorem}
    $\U$ is an isometry from $L^2(\Circ;E)$ to $L^2(\Rl;E)$. 
\end{theorem}
\begin{proof}
    Let $f \in L^2(\Circ;E)$. Then
    \begin{align*}
        \|\U f\|_{L^2(\Rl;E)}^2 &= \int_{\Rl} \frac{1}{\pi|i+t|^2}\|(f\circ\omega^{-1})(t)\|_E^2\;d\lambda\\
        &= \int_\Rl \|(f\circ\omega^{-1})(t)\|_E^2\;d(\omega_*\ha)(t)\\
        &= \int_\Circ \|f\|_E^2 d\ha.
    \end{align*}
    So $\U$ embeds $L^2(\Circ;E)$ into $L^2(\Rl;E)$ isometrically.
    We may similarly prove the opposite embedding.
\end{proof}


We also have,
\begin{theorem}
    $\U L^\infty(\Circ;E) \subset L^\infty(\Rl;E)$. The inclusion
    here is continuous, and it is not true that $L^\infty(\Rl;E) = \U L^\infty(\Circ;E)$.
\end{theorem}
\begin{proof}
    This is evident from the definition of $\U$. Let $f \in L^\infty(\Circ;E)$. Then,
    \begin{align*}
        \|\U f\|_{L^\infty(\Rl;E)} &= \sup_{t \in \Rl} \frac{1}{\sqrt{\pi}|i+t|}\|(f\circ \omega^{-1})(t)\|_{E} \\
        &< \sup_{z \in \Circ} \|f(z)\|_E\\
        &= \|f\|_{L^\infty(\Circ;E)}.
    \end{align*}
    
    However, consider a constant function $c \in L^\infty(\Rl;E)$.
    We see that $\U^{-1}c$ is unbounded.
\end{proof}

A natural question is then to determine $\U^{-1}L^\infty(\Rl;E)$, this will
be answered in section \ref{BMOandVMO}.

Similarly to how $L^\infty(\Circ;E)$ is mapped into $L^\infty(\Rl;E)$, spaces
of continuous and differentiable functions on $\Circ$ are mapped into
continuous and differentiable functions on $\Rl$ but not vice versa.


\section*{Hardy spaces}
We now introduce the crucial Hardy spaces.
\subsection*{Definitions}
Hardy spaces are initially defined as spaces of functions analytic in the unit
ball or the upper half plane.
In particular, for a function $f:\Disc\rightarrow E$
that is complex differentiable, and $p \in (0,\infty)$, we define
\begin{equation*}
    \|f\|_{H^p(\Disc;E)} = \sup_{r \in (0,1)} \left(\int_{\Circ} \|f(r\zeta)\|_E^p\;d\ha(\zeta)\right)^{1/p}.
\end{equation*}
and
\begin{equation*}
    \|f\|_{H^\infty(\Disc;E)} = \sup_{z \in \Disc} \|f(z)\|_E.
\end{equation*}
There is a very similar definition for Hardy spaces on the upper half plane,
if $f:\Half\rightarrow E$ is complex differentiable, we define
\begin{equation*}
    \|f\|_{H^p(\Half;E)} = \sup_{y>0} \left(\int_\Rl \|f(x+iy)\|_E^p\;d\lambda(x)\right)^{1/p}
\end{equation*}
for $p \in (0,\infty)$
and
\begin{equation*}
    \|f\|_{H^\infty(\Half;E)} = \sup_{z \in \Half} \|f(z)\|_E.
\end{equation*} 

Thus, we consider the Hardy spaces $H^p(\Disc;E)$ and $H^p(\Half;E)$
of complex differentiable functions $f$ on $\Disc$ and $g$ on $\Half$
respectively such that $\|f\|_{H^p(\Disc;E)} < \infty$ and $\|g\|_{H^p(\Half;E)} < \infty$
respectively for $p \in (0,\infty]$. 

Typically, we do not consider Hardy spaces as spaces of functions
on the unit ball or upper half plane. This is justified
by the following pair of theorems,
\begin{theorem}
    For almost all $\zeta \in \Circ$, and $f \in H^p(\Disc;E)$, the radial limit
    \begin{equation*}
        \tilde{f}(\zeta) = \lim_{r\rightarrow1^{-}}f(r\zeta).
    \end{equation*}
    exists, defines a function $\tilde{f} \in L^p(\Circ;E)$.
\end{theorem}
\begin{theorem}
    For almost all $x \in \Rl$, and $f \in H^p(\Half;E)$, the limit
    \begin{equation*}
        \tilde{f}(x) = \lim_{y\rightarrow 0^{+}} f(x+iy)
    \end{equation*}
    exists, defines a function $\tilde{f} \in L^p(\Rl;E)$.
\end{theorem}

The subspaces of $L^p(\Circ;E)$ and $L^p(\Rl;E)$ that can be obtained
as boundary values in this way are denoted $H^p(\Circ;E)$ and $H^p(\Rl;E)$.


\section*{Bounded and Vanishing Mean Oscillation}
\label{BMOandVMO}

\begin{thebibliography}{9}
\bibitem{connes94}
     Connes A., 
    \emph{Noncommutative Geometry}
     Academic Press, 
     San Diego, 
     CA, 
     1994
\bibitem{peller}
    Peller V.V.,
    \emph{Hankel Operators and their Applications}
    Springer-Verlag,
    New York, 
    NY,
    2003    
\bibitem{garnett}
    Garnett J.B.
    \emph{Bounded analytic functions}
    Springer-Verlag,
    New York,
    NY,
    2007    
\end{thebibliography}


\end{document}
